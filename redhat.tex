\documentclass[10pt,landscape]{article}
\usepackage{multicol}
\usepackage{calc}
\usepackage{ifthen}
\usepackage[landscape]{geometry}
\usepackage[pdftex]{graphicx}
% To make this come out properly in landscape mode, do one of the following
% 1.
%  pdflatex latexsheet.tex
%
% 2.
%  latex latexsheet.tex
%  dvips -P pdf  -t landscape latexsheet.dvi
%  ps2pdf latexsheet.ps


% If you're reading this, be prepared for confusion.  Making this was
% a learning experience for me, and it shows.  Much of the placement
% was hacked in; if you make it better, let me know...


% 2008-04
% Changed page margin code to use the geometry package. Also added code for
% conditional page margins, depending on paper size. Thanks to Uwe Ziegenhagen
% for the suggestions.

% 2006-08
% Made changes based on suggestions from Gene Cooperman. <gene at ccs.neu.edu>


% To Do:
% \listoffigures \listoftables
% \setcounter{secnumdepth}{0}


% This sets page margins to .5 inch if using letter paper, and to 1cm
% if using A4 paper. (This probably isn't strictly necessary.)
% If using another size paper, use default 1cm margins.
\ifthenelse{\lengthtest { \paperwidth = 11in}}
	{ \geometry{top=.5in,left=.5in,right=.5in,bottom=.5in} }
	{\ifthenelse{ \lengthtest{ \paperwidth = 297mm}}
		{\geometry{top=1cm,left=1cm,right=1cm,bottom=1cm} }
		{\geometry{top=1cm,left=1cm,right=1cm,bottom=1cm} }
	}

% Turn off header and footer
\pagestyle{empty}
 

% Redefine section commands to use less space
\makeatletter
\renewcommand{\section}{\@startsection{section}{1}{0mm}%
                                {-1ex plus -.5ex minus -.2ex}%
                                {0.5ex plus .2ex}%x
                                {\normalfont\large\bfseries}}
\renewcommand{\subsection}{\@startsection{subsection}{2}{0mm}%
                                {-1explus -.5ex minus -.2ex}%
                                {0.5ex plus .2ex}%
                                {\normalfont\normalsize\bfseries}}
\renewcommand{\subsubsection}{\@startsection{subsubsection}{3}{0mm}%
                                {-1ex plus -.5ex minus -.2ex}%
                                {1ex plus .2ex}%
                                {\normalfont\small\bfseries}}
\makeatother

% Define BibTeX command
\def\BibTeX{{\rm B\kern-.05em{\sc i\kern-.025em b}\kern-.08em
    T\kern-.1667em\lower.7ex\hbox{E}\kern-.125emX}}

% Don't print section numbers
\setcounter{secnumdepth}{0}


\setlength{\parindent}{0pt}
\setlength{\parskip}{0pt plus 0.5ex}


% -----------------------------------------------------------------------
%\fbox{ /etc/sysconfig/network-scripts/ifcfg-eth0 }
%\begin{tabular}{@{}ll@{}}
%\verb!DEVICE=!  & "eth0" \\
%\end{tabular}
%\fbox{ /etc/resolv.conf}

\begin{document}

\raggedright
\footnotesize
\begin{multicols}{3}


% multicol parameters
% These lengths are set only within the two main columns
%\setlength{\columnseprule}{0.25pt}
\setlength{\premulticols}{1pt}
\setlength{\postmulticols}{1pt}
\setlength{\multicolsep}{1pt}
\setlength{\columnsep}{2pt}

\begin{center}
     \Large{\textbf{RedHat Cheat Sheet}} \\
\end{center}

\section{Network Device Config}
\begin{itemize}
\item \textbf{/etc/sysconfig/network-scripts/ifcfg-eth0}
\item If BOOTPROTO="dhcp", then DNS,GATEWAY,IPADDR,etc will not be set.
\begin{verbatim}
DEVICE="eth0" 
BOOTPROTO="static" or "dhcp" 
DNS1="*.*.*.*" 
GATEWAY="*.*.*.*" 
NETMASK="*.*.*.*"
IPADDR="*.*.*.*" 
HWADDR="*:*:*:*:*:*"
UUID="*"
ONBOOT="yes" 
TYPE="Ethernet" 
NM_CONTROLLED="no"
\end{verbatim}


\item A virtual interface may be added with the following command:
\begin{verbatim}
ip add *.*.*.*/24 dev eth0 label eth0:0
\end{verbatim}

\item \textbf{/etc/sysconfig/network-scripts/ifcfg-eht0:0}
\begin{verbatim}
DEVICE=eth0:0
IPADDR=*.*.*.*
PREFIX=24
ONBOOT=yes
ONPARENT=yes
NM_CONTROLLED=no
\end{verbatim}
\item \textbf{/etc/sysconfig/network}
\begin{verbatim}
NETWORKING=yes 
HOSTNAME=hostname.domainname.com 
GATEWAY=192.168.1.1
\end{verbatim}
\end{itemize}
\subsection{Network Interface Utils}
\begin{tabular}{@{}ll@{}}
\texttt{ifconfig -a} & Show summary for all interfaces. \\
\texttt{ip addr show} & Show summary for all interfaces. \\
\texttt{mii-tool -v devname} & Show device mode. \\
\texttt{ethtool devname} & Show device negotiated tx speed. \\
\texttt{ethtool -S devname} & Show transmission statistics. \\
\texttt{netstat -i} & Show transmission statistics. \\
\texttt{system-config-network} & Cli tool for managing dev settings. \\
\texttt{service network restart} & Restart network services.  \\
\end{tabular}

\subsection{Nameservice Resolution}
\begin{itemize}
\item  \textbf{/etc/resolv.conf}
\begin{verbatim}
nameserver 8.8.8.8 
domain mydomain.com
search mydomain.com
\end{verbatim}

\item \textbf{/etc/nsswitch.conf}
\begin{verbatim}
hosts & files dns 
\end{verbatim}

\item \textbf{/etc/hosts}
\begin{verbatim}
127.0.0.1  localhost.localdomain
::1  localhost.localdomain 
\end{verbatim}
\end{itemize}

\subsection{Connectivity Utils}
\begin{tabular}{@{}ll@{}}
\texttt{ping -c 5 -R host} & TX 5 packets and record route \\
\texttt{traceroute host} &  network path and record hops. \\
\texttt{mrt host} & trouceroute and hop statistics. \\
\texttt{netstat -r} & Show route tables. \\
\texttt{ip route} & Show route tables. \\
\texttt{getent hosts hostname} & Show nameservice record.  \\
\texttt{host hostname} & Show nameservice record.  \\
\texttt{nslookup hostname} & Show dns nameservice record.  \\
\texttt{dig hostname +noall +answer} & Show dns nameservice record.  \\
\end{tabular}

\subsection{Firewall}
\fbox{ /etc/sysconfig/iptables
/etc/sysconfig/iptables-config}
\begin{tabular}{@{}ll@{}}
\texttt{system-config-firewall} & config tool \\
\texttt{service iptables save} & save ruleset \\
\texttt{service iptables panic} & drop everything \\
\texttt{iptables -vnL CHAIN } & list ruleset for CHAIN \\
\texttt{iptables -Lv --line-numbers} & list ruleset for CHAIN \\
\texttt{iptables -A INPUT } & append to INPUT CHAIN \\
\texttt{iptables -I INPUT } & insert rule at line 1 \\
\texttt{iptables -R CHAIN number} & replace rule number \\
\texttt{iptables -D CHAIN number} & delete rule number \\
\end{tabular}
\begin{verbatim}
iptables -P DROP #change default policy to DROP
iptables -A INPUT -m state --state ESTABLISHED,RELATED -j ACCEPT
iptables -A INPUT REJECT --help
iptables -t nat -A POSTROUTING -o eth1 -j MASQUERADE
iptables -t nat -A POSTROUTING -s 192.168.0.0/24 -j SNAT --to-source 172.12.17.1
iptables -t nat -A PREROUTING -p tcp --dport 80 -j DNAT --to-destination 192.168.0.5
\end{verbatim}
\subsection{Firewall Example Rulesets}
\begin{tabular}{@{}ll@{}}
\texttt{-I INPUT -p tcp -s *.*.*.* --dport 22 -j ACCEPT} & allow 22\\
\texttt{-I INPUT -p tcp -s *.*.*.* --dport 22 -j DROP} & allow 22\\
\end{tabular}

\subsection{TCP Wrappers}
\begin{itemize}
\item \textbf{/etc/hosts.allow}
\begin{verbatim}
    ALL:127.0.0.1
\end{verbatim}
\item \textbf{/etc/hosts.deny}
\begin{verbatim}
    ALL:ALL EXCEPT .example.com
\end{verbatim}
\end{itemize}
\section{Storage}
\fbox{ /etc/fstab}

This file contains 6 columns.  The have the following header descriptions:

\{dev\} \{mntpnt\} \{fstype\} \{mntopts\} \{dumpopt\} \{fsckopt\}

\begin{tabular}{@{}ll@{}}
\texttt{fdisk -cul} & list partitions \\
\texttt{fdisk -cu /dev/sda2} & menu driven partition editor \\
\texttt{mkfs.ext4 /dev/sda2} & make file system ext4 on dev \\
\texttt{partprobe /dev/sda2} & make kernel aware of partition changes \\
\texttt{blkid} & returns the UUID information for disks \\
\texttt{cat /proc/mount} & returns what is mounted \\
\texttt{mount -l} & returns what is mounted \\
\texttt{mount -o remount,rw /} & remounts / rw\\
\texttt{mount -o loop /path/to.iso} & mounts mounts an iso file\\

\end{tabular}

\subsection{Encrypted File Systems}
The following are arguments to the cryptsetup command.
must execute the following:
\begin{enumerate}
\item Initialize Luks Partition
\begin{verbatim}
cryptsetup luksFormat /dev/vdaN
cryptsetup luksOpen /dev/vdaN NAME
mkfs.ext4 /dev/mapper/NAME
\end{verbatim}
\item Mount persistantly
\begin{verbatim}
NAME /dev/vdaN /path/to/passwd > /etc/crypttab
/dev/mapper/NAME /mount ext4 defaults 1 1 >/etc/fstab
cryptsetup luksAddKey /dev/vdaN /path/to/passwd
\end{verbatim}
\end{enumerate}
%touch /fpasskey \&\& chmod 600 /fpasskey  
%echo mypasskey>/fpasskey 

%\begin{tabular}{@{}ll@{}}
%\texttt{luksFormat /dev/sda2} & initialize Luks partition \\
%\texttt{luksOpen /dev/sda2 freeagent} & set up freeagent partition \\
%\texttt{luksAddKey /dev/sda2 /fpasskey} & tell cryptsetup about key  \\
%\end{tabular}

%add the following to /etc/fstab
%/dev/mapper/freeagent /freeagent ext4 \_netdv 1 1
%add the following to /etc/crypttab
%freeagent /dev/sda2 /freeagent


%fdisk /dev/sda
\subsection{Volumes}
\begin{enumerate}
\item create/move/remove a physical volume 
\begin{verbatim}
pvcreate /dev/sda#
\end{verbatim}
\item create/add/remove device in volume group
\begin{verbatim}
vgcreate volume_group_name /dev/sda#
vgextend volume_group_name /dev/sda#
\end{verbatim}
\item create/add/remove logical volume
\begin{verbatim}
lvcreate -n lvname -L #G vgname
lvextend -L12G /dev/volume_group_name/lvname
lvextend -L+1G /dev/volume_group_name/lvname
lvextend -l +100\%FREE /dev/volume_group_name/lvname
\end{verbatim}
\item make a filesystem on the new device
\begin{verbatim}
mkfs -t ext4 /dev/mapper/vgname-lvname
mkfs -t ext4 /dev/vgname/lvname
fsck -f /dev/mapper/vgname-lvname
resize2fs -p /dev/vgname/lvname
\end{verbatim}
\item \textbf{/etc/fstab}
\begin{verbatim}
/dev/mapper/vgname-lvname /mountpoint ext4 defaults 1 2
\end{verbatim}
\end{enumerate}
\begin{tabular}{@{}ll@{}}
\texttt{pvdisplay}                                     & show physical volume info\\
\texttt{pvscan}                                        & show physical volume info (minimal)\\
\texttt{vgdisplay}                                     & show volume group info\\
\texttt{vgscan}                                        & show volume group info (minimal)\\
\texttt{lvdisplay}                                     & show physical volume info\\
\texttt{lvscan}                                        & show logical volume info (minimal)\\
\texttt{lvremove /dev/sda\#}                           & remove logical volume label\\
\texttt{vgreduce vgroup\_name /dev/sda\# dest}         & remove disk from vgroup\\
\texttt{vgremove vgroup\_name}                         & remove the volume group  \\
\texttt{lvreduce -L \#G /dev/mapper/vgname-lvname}     & reduce logical volume\\
\texttt{resize2fs -p /dev/mapper/vgname-lvname \#G}    & reduce file sys size\\
\texttt{mount -a}                                      & mount it all\\
\texttt{umount filesystem}                             & unmount it \\
\end{tabular}

\section{Package Management}

\subsection{Using Yum}
The following are arguments to the yum command:

\begin{tabular}{@{}ll@{}}
\texttt{-nogpgcheck localinstall bla.rpm} & install bla.rpm \\
\texttt{-y bla} & install  \\
\texttt{remove bla} & un-install  \\
\texttt{update bla} & update  \\
\texttt{search bla} & search packages associated with bla \\
\texttt{list installed} & identify install packages \\
\texttt{list available *ftp*} & identify available packages \\
\texttt{provides filename} & determine file association \\
\texttt{grouplist } & identify install groups \\
\texttt{groupinstall 'groupname' } & install group software\\
\texttt{groupremove 'groupname' } & un-install group software\\
\texttt{repolist all } & list all rpm repos\\
\texttt{--enablerepo=reponame install bla} & install from disabled repo \\

\end{tabular}

\subsection{Using rpm}
The following are arguments to the rpm command:

\begin{tabular}{@{}ll@{}}
\texttt{-qlp bla.rpm} & list files installed by bla.rpm  \\
\texttt{-q --requires bla.rpm} & list dependancies \\
\texttt{-q --whatprovides file} & file package association \\
\texttt{-qd} & list documentation \\
\texttt{-qc} & list config files\\
\texttt{-qa} & list installed pkgs \\
\texttt{-i bla.rpm} & install bla.rpm  \\
\texttt{-q -i bla.rpm} & get into on bla.rpm  \\
\texttt{-U bla.rpm} & update bla.rpm  \\
\texttt{-e bla} & un-install  \\
\end{tabular}
\subsection{Building RPMs}
yum install rpm-build rpmdevtools
\begin{tabular}{@{}ll@{}}
\texttt{rpmdev-setuptree} & creates rpmbuild directories \\
\texttt{rpmdev-newspec} & creates a new spec file \\
\texttt{rpmbuild -ba specfile} & build rpm \\
\texttt{rpm -i bla.rpm} & install bla.rpm \\
\texttt{rpm -e bla.rpm} & un-insdocumentation tall bla.rpm \\
\end{tabular}
procedure:
\begin{enumerate}
\item yum install rpm-build rpmdevtools
\item rpmdev-setuptree
\item mkdir $\sim$/software-1.0
\item touch $\sim$/software-1.0/configure;chmod u+x configure
\item touch $\sim$/software-1.0/sfile
\item tar -czf software1.0-1.el.x86\_64.tar.gz software-1.0
\item cp software-1.0.tar.gz $\sim$/build/SOURCES
\item vi $\sim$/rpmbuild/SPECS/software-1.0.spec
\end{enumerate}
SPECFILE:
\begin{verbatim}
Source0:software-1.0-1.el.x86_64.tar.gz
%build
#make %{?_smp_mflags}
%install
#make install DESTDIR=%{buildroot}
install -d -m 0755 $RPM_BUILD_ROOT/d_dir
install -m 0644 $RPM_BUILD_ROOT/d_dir/sfile
%files
/d_dir/sfile
\end{verbatim}
rpmbuild directory structure
sections of specfile
yum repository
\subsection{YUM Repo}
\begin{enumerate}
\item edit /etc/yum.repos.d/your.repo
\begin{verbatim}
 [repositoryid]
 name=somename
 baseurl=http://path.to.repo
 gpgcheck=0
\end{verbatim}
\end{enumerate}
\section{KVM}
\begin{tabular}{@{}ll@{}}
\texttt{virsh list --all} & identify guest hosts \\
\texttt{virsh start host} & "power on" host\\
\texttt{virsh destroy host} & "power off" host\\
\texttt{virsh shutdwon host} & shut it down nicely\\
\texttt{virsh autostart host} & "power on" at boot\\
\texttt{virt-install --prompt} & "menu" driven install\\
\texttt{virt-viewer host} & open console \\
\texttt{virt-manager} & starts the gui hypervisor\\
\end{tabular}

\section{SELINUX}
\begin{tabular}{@{}ll@{}}
\texttt{sestatus} & identify security status \\
\texttt{chcon} & change the context for a file \\
\texttt{getenforce} & display selinux mode\\
\texttt{setenforce permissive} & change enforcement of selinux\\
\texttt{setenforce enforcing} & change enforcement of selinux \\
\texttt{ls -Z file} & show file's context \\
\texttt{getsebool -a} & display selinux boolean \\
\texttt{seaudit} & open gui to view selinux messages\\
\texttt{restorecon -Rv} & restore default file context\\
\texttt{system-config-selinux} & launch the SELinux admin tool\\
\texttt{ps -eZ} & view process context information\\

\end{tabular}
yum install policycoreutils-gui setools-gui setroubleshoot-server
/etc/sysconfig/selinux
\section{Kickstart}
\begin{verbatim}
ksvalidator
\end{verbatim}
\begin{tabular}{@{}ll@{}}
\texttt{system-config-kickstart} & launch kickstart gui\\
\end{tabular}
\section{Services}
\begin{tabular}{@{}ll@{}}
\texttt{service foo restart} & restarts the foo service \\
\texttt{service -} & restarts the foo service \\
\texttt{chkconfig foo on} & start foo on reboot\\
\end{tabular}
\subsection{httpd}
\begin{enumerate}
\item yum groupinstall "Web Server"
\item yum install elinks
\end{enumerate}
/etc/pki/tls
/etc/httpd/conf.d/ssl.conf
/etc/httpd/conf/httpd.conf
\begin{verbatim}
NameVirtualHost 192.168.0.196:80
<VirturalHost 192.168.0.196:80>
 ServerAdmin webmaster@example.com
 DocumentRoot /var/www/docs/123xxx
 ServerName 123xxx.com
</VirtualHost>
<VirturalHost 192.168.0.196:80>
 ServerAdmin webmaster@example.com
 DocumentRoot /var/www/docs/xxx123
 Servername xxx123.com
</VirtualHost>
\end{verbatim}
\begin{enumerate}
\item put cgi script in /var/www/cgi-bin
\item getsebool -a |grep cgi
\item setsebool httpd\_enable\_cgi on
\item cd /var/www/docroot/
\item ln -s ../../cgi-bin
%/etc/httpd/conf/httpd.conf
\item     remove scriptalias and directory stanza for CGI
\item service httpd restart
\end{enumerate}
\subsection{DNS}
\begin{enumerate}
\item yum install named
\item edit /etc/named.conf
\begin{verbatim}
    listen-on 53 { 127.0.0.1;192.168.0.196;};
    allow-query {localhost;192.168.0.0/24;};
    forwarders {nameserverip;}
\end{verbatim}
\item service named start
\item chkconfig named on
\item open firewall port
\end{enumerate}
\subsection{NFS}
\begin{enumerate}
\item yum -y install nfs*
\item edit /etc/exports
\begin{verbatim}
    /shared *.exampel.com(rw)
\end{verbatim}
\item exportfs -va
\item .....open firewall
\end{enumerate}
\subsection{AutoFs}
\begin{enumerate}
\item yum -y install autofs
\item edit /etc/auto.master
\begin{verbatim}
    /misc/shared    /etc/auto.misc
\end{verbatim}
\item edit /etc/auto.misc
\begin{verbatim}
    shared -fstype=nfs,rw   server:/shared
\end{verbatim}
\item service autofs restart
\item cd /misc/shared
\end{enumerate}
\subsection{smtp}
\begin{enumerate}
\item yum install postfix
/var/log/maillog
\item edit /etc/postfix/main.cf:
\begin{verbatim}
myhostname=servername.example.com
mydomain=example.con
myorign = $mydomain
inet_interfaces=all
mydestination=$myhostname,localhost.$mydomain,localhost,$mydomain
mynetworks = 127.0.0.0/8 10.0.0.0/24
relay_domains =$mydestination
#relayhost = $mydomain ; only if not internet
\end{verbatim}
\item service postfix start
\item chkconfig postfix on
\item open firewall
\end{enumerate}
\subsection{vsftpd}
\begin{enumerate}
\item yum install vsftpd
\item edit /etc/vsftpd/vsftpd.conf
\begin{verbatim}
    anonymous_enable=YES
\end{verbatim}
\item service vsftpd start
\item chkconfig vsftpd on
\item open firewall
\end{enumerate}
\subsection{Samba}
\begin{enumerate}
\item yum install samaba
\item edit /etc/samba/samba.conf
\begin{verbatim}
    workgroup=SMBGROUP
    hosts allow=127. 192.168.0.

    [shared]
    comment=SambaShare
    path=/shared
    browseable=yes
    public=yes
    writable=no
    printable=no
    valid users=%U or +GROUP
\end{verbatim}
\item create user
\begin{verbatim}
smbpasswd -a user
\end{verbatim}
\item restart the smb service
\item mount the cifs share
\begin{verbatim}
mount cifs //server/shared/test -o username=user
To make persistant:
cat /etc/mtab|grep relevant string >> /etc/fstab
\end{verbatim}
\item open firewall
\end{enumerate}
\section{Manage Users}
\begin{tabular}{@{}ll@{}}
\texttt{useradd} & adds a user\\
\texttt{usermod} & modify user properties\\
\texttt{chage} & password aging\\
\end{tabular}
\section{Shell}

\begin{tabular}{@{}ll@{}}
\texttt{fdisk -cul} & list partitions \\
\end{tabular}
\subsection{Structures}

\begin{verbatim}
------------------------
if [ "$val" = "$val"]; then
    echo stuff
else
    more stuff
fi
------------------------
c=0
while [ $c -lt 10 ]; do
    echo stuff
    let c=c+1
done
------------------------
for val in \{5..10\}
do
done
------------------------
\end{verbatim}
%cp /file.txt{,-old}
%mkdir -p path/{dir1,dir2}
%ps -ef|tail -1|awk '{print $2}'
%ehco foo bar
%\^ehco\^echo
%rename 's/oldname/newname/'*.ext
%!?foo?:p search history for foo and print
%!?foo executes last command containing foo
%pstree
%pidof service
%od -c
%stat
%strings -a
%CDPATH='.:~:/usr/local/apache/htdocs:/disk1/backups'
%sed -i 's/old-word/newword/g' *.txt
%sed -i 1d foo.txt
%sed -i -e \'/\^\$/d\' foo.txt
%sed -i -e \'/\#.*\$/d\' foo.txt
%--
\section{Files}
\begin{tabular}{@{}ll@{}}
\texttt{chmod 1757} & sets sticky bit\\
\texttt{chmod 2750} & sets sgid \\
\texttt{chmod 4750} & sets suid\\
\texttt{setfacl -m u:lisa:r file} & sets acl\\
\texttt{setfacl -m u:lisa:- file} & sets acl no nothing\\
\texttt{setfacl -m m::rx file} & sets acl mask\\
\texttt{setfacl -x g:groupname file} & remove group access\\
\texttt{setfacl -b file} & remove acls \\
\end{tabular}
%$ setfacl -m u:lisa:r file
%
%$ getfacl file1 > acls.txt
%$ setfacl -f acls.txt file2
%
%$ setfacl -m m::rx file, effective mask remove write
%$ setfacl -x g:staff file, remove group access
%$ setfacl -m d:u::rwx,d:g::rwx,d:m:rwx,d:o:r-x groupshare
%setfacl -b file, clear acls for file
\section{Kernel}
\begin{verbatim}
sysctl -a list tuned variables
sysctl -a list tuned variables
sysctl -w modify a variable
sysctl -w modify a variable
sysctl -p reread sysctl.conf
sysctl -p reread sysctl.conf

/etc/sysctl.conf
    net.ip4.ip\_forward=1 , enable packet forwarding
sysctl -w modify a variable
sysctl -p reread sysctl.conf
sysctl -p reread sysctl.conf

/etc/sysctl.conf
\end{verbatim}
\section{Github}
\begin{verbatim}
git remote add upstream git@github.com:username/project.git
git fetch upstream
git clone git@github.com:username/project.git
git config --global user.name username
git config --global user.email emailaddress
ssh-keygen -t rsa -C emailaddress
ssh -T git@github.com

\end{verbatim}
\end{multicols}
\end{document}
